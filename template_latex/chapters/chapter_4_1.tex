\section{Υλοποίηση {\lt Gauss} με Παράλληλο Προγραμματισμό στη {\lt CPU (C/POSIX Threads)}}

%%%%%%%%%%%%%%%%%%%%%%%%%
\subsection{Εισαγωγή}


%%%%%%%%%%%%%%%%

\subsubsection{Μετάβαση από τον σειριακό στον παράλληλο προγραμματισμό.}
    
\noindent
Κατά τη διάρκεια των 70ς, 80ς και ένα μέρος των 90ς, ο σειριακός προγραμματισμός (δηλαδή η χρήση ενός {\lt thread}) ήταν το πιο διάσιμο είδος προγραμματισμού που χρησιμοποιούσε τη {\lt CPU} (κεντρική μονάδα επεξεργασίας) για την εκτέλεση των προγραμμάτων. Γράφαμε ένα πρόγραμμα το οποίο έκανε μία διεργασία και έδινε αποτέλεσμα.
